\usepackage{ifthen}
\usepackage{subcaption}
\usepackage{dsfont}
\usepackage{graphicx}
\usepackage{relsize}
\usepackage{exscale}
\usepackage[dvipsnames]{xcolor}
% Indentacion para que vaya hacia adentro
\newenvironment{IndentedBlock}
{
    \begin{list}{}{
        \leftmargin=2em
        \rightmargin=0em
        \topsep=0pt
        \partopsep=0pt
        \parsep=3pt
        \itemsep=0pt
    }
    \item\relax
}
{
    \end{list}
}
%
%

% para los tres comandos de parametros
% #1 es el nombre
% #2 es el tipo
\newcommand{\In}[2]{\textsf{in} #1 : #2}
\newcommand{\Out}[2]{\textsf{out} #1 : #2}
\newcommand{\Inout}[2]{\textsf{inout} #1 : #2}

\newcommand{\requiere}[2][]{%
    {\normalfont\bfseries\ttfamily requiere}%
    \ifthenelse{\equal{#1}{}}{}{\ {\normalfont\ttfamily #1}}%
    \ :\ %
    \{\ensuremath{#2}\}\\ %
}

\newcommand{\asegura}[2][]{%
    {\normalfont\bfseries\ttfamily asegura}%
    \ifthenelse{\equal{#1}{}}{}{\ {\normalfont\ttfamily #1}}%
    \ :\ %
    \{\ensuremath{#2}\}\\ %
}

%
%
%
%
%
% custom
%
%



\newcommand{\predheader}[2]{%
{\normalfont\bfseries\ttfamily\noindent pred}
{\normalfont\ttfamily #1}%
(#2)\ %
}

% #1 nombre del pred.
% #2 parametros del pred
\newenvironment{pred}[2]{
\predheader{#1}{#2}%
\{%
\begin{IndentedBlock}
}{
\end{IndentedBlock}

\}%
}

\newcommand{\auxheader}[3]{
{\normalfont\bfseries\ttfamily aux}
{\normalfont\ttfamily #1}%
(#2) : {\normalfont\ttfamily #3}%
}

% #1 nombre del aux
% #2 parametros del aux
% #3 tipo del aux
\newenvironment{aux}[3]{
\auxheader{#1}{#2}{#3}\ = $\cdots$%
\vspace{-1pt}
\par
\begin{IndentedBlock}
}{
\end{IndentedBlock}
\vspace{0em}
}

\newcommand{\procheader}[3]{
{\normalfont\bfseries\ttfamily proc}
{\normalfont\ttfamily #1}%
(#2)%
\ifthenelse{\equal{#3}{}}{\ }{ : #3\ }%
}

% #1 nombre del proc.
% #2 parametros
% #3 tipo del resultado
\newenvironment{proc}[3]{
\procheader{#1}{#2}{#3}%
\{%
\begin{IndentedBlock}
}{
\end{IndentedBlock}
\}%
}

\usepackage{ifthen}

\newcommand{\tadheader}[2]{
{\normalfont\bfseries\ttfamily\noindent TAD}%
\ %
{\normalfont\ttfamily #1}%
\ifthenelse{\equal{#2}{}}{}{%
{$\langle$#2$\rangle$}%
}%
}
\newenvironment{tad}[2]{
\tadheader{#1}{#2}
    \{%
    \begin{IndentedBlock}
}{
    \end{IndentedBlock}
    \}%
}

% Tipos
\newcommand{\tnat}{\ensuremath{\mathds{N}}}
\newcommand{\tint}{\ensuremath{\mathds{Z}}}
\newcommand{\treal}{\ensuremath{\mathds{R}}}
\newcommand{\tbool}{\ensuremath{\mathsf{Bool}}}
\newcommand{\tfloat}{\ensuremath{\mathsf{Float}}}
\newcommand{\tchar}{\ensuremath{\mathsf{Char}}}
\newcommand{\tstring}{\ensuremath{\mathsf{String}}}
%tipo generico
\newcommand{\tg}[1]{\ensuremath{\mathrm{#1}}}

%Tipos complejos
\newcommand{\ttuple}[1]{\ensuremath{\mathsf{Tuple}\langle#1\rangle}}
\newcommand{\ttup}[1]{\ttuple{#1}}
\newcommand{\tstruct}[1]{\ensuremath{\mathsf{Struct}\langle#1\rangle}}
\newcommand{\tseq}[1]{\ensuremath{\mathsf{Seq}\langle#1\rangle}}
\newcommand{\tconj}[1]{\ensuremath{\mathsf{Conj}\langle#1\rangle}}
\newcommand{\tdict}[1]{\ensuremath{\mathsf{Dict}\langle#1\rangle}}

% Comentarios para mas claridad
\newcommand{\comment}[1]{%
\textcolor{Periwinkle}{// #1}\\%
}

%suma y union
%#1 limite inferior
%#2 limite superior
\newcommand{\bigsum}[2]{
\ensuremath{\mathlarger{\mathlarger{\mathlarger{\sum}}}\limits_{#1}^{#2}}
}

%#1 limite inferior
%#2 limite superior
\newcommand{\bigunion}[2]{
\ensuremath{\mathlarger{\mathlarger{\mathlarger{\bigcup}}}\limits_{#1}^{#2}}
}

% Parentesis
% en caso de necesitar y hacerlo mas legible
\newcommand{\bbpar}[1]{\biggl(#1\biggr)}
\newcommand{\bpar}[1]{\Bigl(#1\Bigr)}


% Logica
\newcommand{\cand}{\land _L}
\newcommand{\cor}{\lor _L}
\newcommand{\impl}{\ensuremath{\longrightarrow}}
\newcommand{\simpl}{\ensuremath{\rightarrow}} %mas chico
\newcommand{\cimpl}{\ensuremath{\impl _L}}
\newcommand{\csimpl}{\ensuremath{\simpl _L}} %mas chico
\newcommand{\True}{\ensuremath{\mathrm{true}}}
\newcommand{\False}{\ensuremath{\mathrm{false}}}
\newcommand{\Iff}{\ensuremath{\longleftrightarrow}}
\newcommand{\IfThenElseFi}[3]{\ensuremath{\mathsf{if}\ #1\ \mathsf{then}\ #2\ \mathsf{else}\ #3\ \mathsf{fi}}}


\newcommand{\IfThenElse}[3]{%
\mathrm{IfThenElse}(#1, #2, #3)%
}

\newcommand{\nsum}[1][1.4]{% only for \displaystyle
    \mathop{%
        \raisebox
            {-#1\depthofsumsign\depthofsumsign}
            {\scalebox
                {#1}
                {$\displaystyle\sum$}%
            }
    }
}

% Extras

% alias de tipo
% #1 es el nombre del alias
% #2 es el tipo.
\newcommand{\alias}[2]{\noindent #1 \textbf{ES}\ #2}

% observadores
% #1 nombre del tipo
% #2 tipo
\newcommand{\obs}[2]{\textbf{obs} \ensuremath{#1} : #2}
% llamado a funcion
% ej call{foo}{bar}
\newcommand{\call}[2]{\mathrm{#1}(#2)}
