\documentclass{article}
\usepackage{ifthen}
\usepackage{subcaption}
\usepackage{dsfont}
\usepackage{graphicx}
\usepackage{relsize}
\usepackage{exscale}
\usepackage[dvipsnames]{xcolor}
% Indentacion para que vaya hacia adentro
\newenvironment{IndentedBlock}
{
    \begin{list}{}{
        \leftmargin=2em
        \rightmargin=0em
        \topsep=0pt
        \partopsep=0pt
        \parsep=3pt
        \itemsep=0pt
    }
    \item\relax
}
{
    \end{list}
}
%
%

% para los tres comandos de parametros
% #1 es el nombre
% #2 es el tipo
\newcommand{\In}[2]{\textsf{in} #1 : #2}
\newcommand{\Out}[2]{\textsf{out} #1 : #2}
\newcommand{\Inout}[2]{\textsf{inout} #1 : #2}

\newcommand{\requiere}[2][]{%
    {\normalfont\bfseries\ttfamily requiere}%
    \ifthenelse{\equal{#1}{}}{}{\ {\normalfont\ttfamily #1}}%
    \ :\ %
    \{\ensuremath{#2}\}\\ %
}

\newcommand{\asegura}[2][]{%
    {\normalfont\bfseries\ttfamily asegura}%
    \ifthenelse{\equal{#1}{}}{}{\ {\normalfont\ttfamily #1}}%
    \ :\ %
    \{\ensuremath{#2}\}\\ %
}

%
%
%
%
%
% custom
%
%



\newcommand{\predheader}[2]{%
{\normalfont\bfseries\ttfamily\noindent pred}
{\normalfont\ttfamily #1}%
(#2)\ %
}

% #1 nombre del pred.
% #2 parametros del pred
\newenvironment{pred}[2]{
\predheader{#1}{#2}%
\{%
\begin{IndentedBlock}
}{
\end{IndentedBlock}

\}%
}

\newcommand{\auxheader}[3]{
{\normalfont\bfseries\ttfamily aux}
{\normalfont\ttfamily #1}%
(#2) : {\normalfont\ttfamily #3}%
}

% #1 nombre del aux
% #2 parametros del aux
% #3 tipo del aux
\newenvironment{aux}[3]{
\auxheader{#1}{#2}{#3}\ = $\cdots$%
\vspace{-1pt}
\par
\begin{IndentedBlock}
}{
\end{IndentedBlock}
\vspace{0em}
}

\newcommand{\procheader}[3]{
{\normalfont\bfseries\ttfamily proc}
{\normalfont\ttfamily #1}%
(#2)%
\ifthenelse{\equal{#3}{}}{\ }{ : #3\ }%
}

% #1 nombre del proc.
% #2 parametros
% #3 tipo del resultado
\newenvironment{proc}[3]{
\procheader{#1}{#2}{#3}%
\{%
\begin{IndentedBlock}
}{
\end{IndentedBlock}
\}%
}

\usepackage{ifthen}

\newcommand{\tadheader}[2]{
{\normalfont\bfseries\ttfamily\noindent TAD}%
\ %
{\normalfont\ttfamily #1}%
\ifthenelse{\equal{#2}{}}{}{%
{$\langle$#2$\rangle$}%
}%
}
\newenvironment{tad}[2]{
\tadheader{#1}{#2}
    \{%
    \begin{IndentedBlock}
}{
    \end{IndentedBlock}
    \}%
}

% Tipos
\newcommand{\tnat}{\ensuremath{\mathds{N}}}
\newcommand{\tint}{\ensuremath{\mathds{Z}}}
\newcommand{\treal}{\ensuremath{\mathds{R}}}
\newcommand{\tbool}{\ensuremath{\mathsf{Bool}}}
\newcommand{\tfloat}{\ensuremath{\mathsf{Float}}}
\newcommand{\tchar}{\ensuremath{\mathsf{Char}}}
\newcommand{\tstring}{\ensuremath{\mathsf{String}}}
%tipo generico
\newcommand{\tg}[1]{\ensuremath{\mathrm{#1}}}

%Tipos complejos
\newcommand{\ttuple}[1]{\ensuremath{\mathsf{Tuple}\langle#1\rangle}}
\newcommand{\ttup}[1]{\ttuple{#1}}
\newcommand{\tstruct}[1]{\ensuremath{\mathsf{Struct}\langle#1\rangle}}
\newcommand{\tseq}[1]{\ensuremath{\mathsf{Seq}\langle#1\rangle}}
\newcommand{\tconj}[1]{\ensuremath{\mathsf{Conj}\langle#1\rangle}}
\newcommand{\tdict}[1]{\ensuremath{\mathsf{Dict}\langle#1\rangle}}

% Comentarios para mas claridad
\newcommand{\comment}[1]{%
\textcolor{Periwinkle}{// #1}\\%
}

%suma y union
%#1 limite inferior
%#2 limite superior
\newcommand{\bigsum}[2]{
\ensuremath{\mathlarger{\mathlarger{\mathlarger{\sum}}}\limits_{#1}^{#2}}
}

%#1 limite inferior
%#2 limite superior
\newcommand{\bigunion}[2]{
\ensuremath{\mathlarger{\mathlarger{\mathlarger{\bigcup}}}\limits_{#1}^{#2}}
}

% Parentesis
% en caso de necesitar y hacerlo mas legible
\newcommand{\bbpar}[1]{\biggl(#1\biggr)}
\newcommand{\bpar}[1]{\Bigl(#1\Bigr)}


% Logica
\newcommand{\cand}{\land _L}
\newcommand{\cor}{\lor _L}
\newcommand{\impl}{\ensuremath{\longrightarrow}}
\newcommand{\simpl}{\ensuremath{\rightarrow}} %mas chico
\newcommand{\cimpl}{\ensuremath{\impl _L}}
\newcommand{\csimpl}{\ensuremath{\simpl _L}} %mas chico
\newcommand{\True}{\ensuremath{\mathrm{true}}}
\newcommand{\False}{\ensuremath{\mathrm{false}}}
\newcommand{\Iff}{\ensuremath{\longleftrightarrow}}
\newcommand{\IfThenElseFi}[3]{\ensuremath{\mathsf{if}\ #1\ \mathsf{then}\ #2\ \mathsf{else}\ #3\ \mathsf{fi}}}


\newcommand{\IfThenElse}[3]{%
\mathrm{IfThenElse}(#1, #2, #3)%
}

\newcommand{\nsum}[1][1.4]{% only for \displaystyle
    \mathop{%
        \raisebox
            {-#1\depthofsumsign\depthofsumsign}
            {\scalebox
                {#1}
                {$\displaystyle\sum$}%
            }
    }
}

% Extras

% alias de tipo
% #1 es el nombre del alias
% #2 es el tipo.
\newcommand{\alias}[2]{\noindent #1 \textbf{ES}\ #2}

% observadores
% #1 nombre del tipo
% #2 tipo
\newcommand{\obs}[2]{\textbf{obs} \ensuremath{#1} : #2}
% llamado a funcion
% ej call{foo}{bar}
\newcommand{\call}[2]{\mathrm{#1}(#2)}

\setlength{\parskip}{1em}


\begin{document}

\section{Macros TAD, proc, pred y aux}
% Comentarios para escribir sobre tu spec.
\comment{Macros para TAD y spec.}


% aliasing de tipos
\alias{Bloque}{\tstruct{id : \tint, lista : \tseq{\cdots}}}

\par
% Se abre y se cierra el TAD, se puede dejar vacio sus parametros.
\begin{tad}{nombreTad}{}

\end{tad}

\par
% El parametro son los genericos, es en modo texto.
\begin{tad}{Diccio}{K, V}
\obs{d}{\tdict{K, V}}

% Definir procs
\begin{proc}{definirYValidar}{\Inout{D}{Diccio}, \In{k}{K}, \In{v}{V}}{\tbool}
% tambien sirve \requiere{}
\requiere{D = D_0}
% tambien sirve \asegura{}
\asegura{D.d = \call{setKey}{D_0, k, v}}
\asegura{res = \True \Iff k \in D.d}

\end{proc}

% Definir predicados.
\begin{pred}{esLlave}{D : Diccio, k : K}
% poner entrada entre $...$ para escribir en modo matematico en vez de modo texto.
$res = \True \Iff k \in D.d$
\end{pred}

\begin{pred}{soloValoresRepetidos}{D : Diccio}
% Si no le pongo $$ esta en modo texto
Devuelve true si los valores son todos iguales...
\end{pred}

\begin{aux}{sumaElementosSiSonPrimos}{s : \tseq{\tint}}{\tint}
% Al igual que pred hay que entrar a modo math con $$
% Ojo, escribir IfThenElse con las mayusculas, el minusculas es de latex
$\bigsum{i=0}{|s|-1} \IfThenElse{\call{esPrimo}{s[i]}}{s[i]}{0}$
\end{aux}
\end{tad}

\pagebreak
\section{Ejemplo de comandos}
\alias{nombre}{definicion}\\%
\obs{var}{Tipo}\\%
\requiere{\cdots}%
\asegura{\cdots}%
\In{a}{TipoA} \\%
\Out{b}{TipoB} \\%
\Inout{c}{TipoC} \\%
% Tipos
\comment{Tipos}
\tnat, \tint, \treal \\%
\tbool, \tfloat, \tchar, \tstring \\%
% Tipo generico, se usa para escribir en modo math y que salga derecha la letra y no curva.
$\tg{T}$, $T$ \\%
%
% Tipos complejos, son los que tienen parametros
\ttuple{\tint, \tg{T}} \\ % Tambien podes usar \ttup{}
\tstruct{id : \tint, sueldo : \treal} \\%
\tseq{\tchar} \\%
\tconj{\cdots} \\%
\tdict{\tg{K, V}} \\[2em]
\comment{Parentesis grandes (para sumas y uniones sirve)}
$\bbpar{\bpar{\cdots}}$\\
% Suma y union, y otros
\comment{Suma, union y otros}
% las cosas matematica van en modo math osea $$
$\bigsum{i=0}{\infty} \cdots$\\
$\bigsum{i=0}{|A|-1}\bbpar{\bigsum{j=0}{|A[i]|-1} A[i][j]}$\\
$\bigsum{a \in A}{} \call{func}{a}$\\
% Uniones
$\bigunion{subindice}{superindice} aplicacion$\\[2em]
\comment{Llamado a funciones, sirve para el modo math}
% Estos llamados a funciones conservan la tipografia del modo texto pero en modo matematico
$\call{funcion}{a, b}$\\
versus:\\
$funcion(a, b)$\\[2em]
\comment{Logica}
$\True, \False, \land, \lor, \impl, \Iff$\\
$\cand, \cor, \cimpl$\\
\comment{Version corta de logica (por si esta muy grande)}\\
$\simpl y \csimpl$\\
\comment{Dos versiones de IfThenElse}\\
$\IfThenElse{B}{S_1}{S_2}$\\
$\IfThenElseFi{B}{S_1}{S_2}$\\
\end{document}
