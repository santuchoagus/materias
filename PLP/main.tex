\documentclass{article}

\usepackage{graphicx} % Required for inserting images
\usepackage[margin=1in]{geometry}
\usepackage{multicol}
\usepackage{amsmath}
\usepackage{amssymb}
\usepackage{amsfonts}
\usepackage{bm}
\usepackage{mathrsfs}

%\setlength{\columnseprule}{1pt}
\usepackage{fancyvrb}
\newenvironment{haskcode}
{\VerbatimEnvironment\begin{small}\begin{Verbatim}}
{\end{Verbatim}\end{small}\vspace{10pt}}


\setlength{\parindent}{0pt}
\setlength{\parskip}{10pt}

\begin{document}
\section{Funciones conocidas}


\begin{multicols}{2}
\begin{haskcode}
foldr :: (a -> b -> b) -> b -> [a] -> b
foldr f z [] = z
foldr f z (x:xs) = f x (foldr f z xs)
\end{haskcode}

\begin{haskcode}
foldl ... (tambien version con foldr)
\end{haskcode}

\begin{haskcode}
recr :: (a -> [a] -> b -> b) -> b -> [a] -> b
recr f z [] = z
recr f z (x:xs) = f x xs (recr f z xs)
\end{haskcode}

\begin{haskcode}
reverse :: [a] -> [a] -> [a]
reverse [] = []
reverse (x:xs) = reverse xs ++ [xs]
-- Def. alternativa reverse
reverse = foldr (\\x rec -> rec ++ (x:[])) [] 
\end{haskcode}

\begin{haskcode}
map :: (a -> b) -> [a] -> [b]
map f [] = []
map f (x:xs) = f x : (map f xs)
\end{haskcode}

\begin{haskcode}
... version de map con foldr
\end{haskcode}


%\columnbreak

\end{multicols}

\section{Esquemas de recursión}
\subsection{Recursión estructural}
Sea $g :: [a] \to b$
\begin{align}
g \emptyList &= \langle CB\rangle(z) \\
g\ (x:xs) &= \langle CR\rangle(... x\ ...\ (g\ xs) ...)
\end{align}

Los casos base $CB_i$ devuelven $z_i$ que no dependen de $g$, mientras que los casos recursivos no tienen las variables $g$ ni $xs$ por separado, excepto en la expresión $(g\ xs)$.
Podría aparecer de la forma $g\ ...\ xs\ ...$ en caso que g tomara parámetros en cierto orden, pero nunca de la forma $g\ (f \ xs)$ (es decir, operar sobre el contenido de xs antes de llamar a la recursión) ya que esto no garantiza que la lista es cada vez mas pequeña.
\par\noindent
Foldr abstrae el esquema de recursión estructural, y decimos que toda recursión estructural es una instancia de foldr.

\subsection{Recursión primitiva}
Sea $g :: [a] \to b$
\begin{align}
g \emptyList &= \langle CB\rangle(z) \\
g\ (x:xs) &= \langle CR\rangle(... x\ ...\ \bm{xs}\ ...\ (g\ xs) ...)
\end{align}

\par\noindent
Análogo al esquema de recursión estructural, pero permite referirse a $xs$ fuera del término $(g\ xs)$.

\par\noindent
Toda recursión primitiva es una instancia de \textbf{recr}.

\subsection{Recursión iterativa}
Sea $g :: b \to [a] \to b$
\begin{align}
g\ ac \emptyList &= \langle CB\rangle(z) \\
g\ ac\ (x:xs) &= \langle CR\rangle(... x\ ...\ xs\ ...\ \bm{(g\ ac'\ xs)} ...)
\end{align}
\par\noindent
Invoca a $(g\ ac'\ xs)$ donde $ac' = f\  (...ac\ ...x)$ es el nuevo acumulador actualizado en función de su valor anterior y la variable $x$.
\par\noindent
Toda recursión iterativa es una instancia de \textbf{foldl}.

\begin{align*}
foldr\ (\boxplus)\ z\ [a,b,c] &= a\ \boxplus\ (b \boxplus\ (c \boxplus\ z)) \\
foldl\ (\boxplus)\ z\ [a,b,c] &= ((z  \boxplus\ a)  \boxplus\ b) \boxplus\ c
\end{align*}


\section{Tipos de datos algebráicos}
\subsection{Tipos enumerados}
Constructores de tipo no extensibles, y declara que son los únicos constructores de ese tipo.

\begin{centeredhaskcode}
data Tipo = Const1 | Const2 | Const3 | ...
\end{centeredhaskcode}

\subsection{Tipos producto}
\begin{centeredhaskcode}
data Tipo = Constuctor T1 T2 T3
\end{centeredhaskcode}
Donde T1 T2 y T3 son tipos, es decir un constructor con muchos parametros.

\subsection{Tipos recursivos}
Son tipos de datos inductivos, tiene un constructor base "CBase" y un constructor que toma al menos un parametro del tipo que lo está definiendo.
\begin{centeredhaskcode}
data Tipo = CBase | Constructor Tipo
\end{centeredhaskcode}

\subsection{Caso general}
Los tipos algebráicos tienen la forma
\begin{align*}
\mathrm{data\ }T & = CB_1 \langle cb\_params_1\ \text{sin el tipo } T \rangle\\  
&| \quad\quad \vdots \\
&|\quad  CB_1 \langle cb\_params_n\ \text{sin el tipo } T\ \rangle\\
&|\quad  CR_1 \langle cr\_params_1 \dots\ T \rangle\\
&| \quad\quad \vdots \\
&|\quad  CR_1 \langle cr\_params_m \dots\ T \rangle\\
\end{align*}

\subsection{Recursion estructural en general}
Sea $g :: T \to \mathbb{Y}$
\begin{align*}
g\ (CB_1\ \langle cb_\_params_1\rangle)\ \dots&= \langle CB_1\rangle(z_1) \\
& \hspace{5pt}\vdots\\
g\ (CB_n\ \langle cb_\_params_n\rangle)\ \dots&= \langle CB_n\rangle(z_n) \\
g\ (CR_2\ \langle cr_\_params_1\rangle)\ \dots&= \langle CR_1\rangle(z_1) \\
& \hspace{5pt}\vdots\\
g\ (CR_m\ \langle cr_\_params_m\rangle)\ \dots&= \langle CR_m\rangle(z_m    ) \\
\end{align*}
Donde $CB_k$, $CR_k$ es el $k$-iésimo caso base/recursivo, y $cb\_params_k$, $cr\_params_k$ los parametros para ese constructor respectivamente.

Así como definiamos el fold en listas de tal forma que el tipo es
\begin{haskcode}
foldr :: (a -> b -> b) -> b -> [a] -> b
\end{haskcode}
Esto significa que \verb|(a -> b -> b)| es la función que va a manejar el constructor cons de listas \verb+Cons List a+, y el \verb|b| que aparece solo le corresponde al caso base de la lista vacía$\emptyList$.

De esta podemos definir la función fold para cualquier tipo recursivo armando una función por cada uno de sus constructores.

[TODO: falta poner mas info acá o un ejemplo]

\section{Inducción estructural}
\subsection{Principio de inducción estructural}
\begin{samepage}
Sea $\mathcal{P}(y)$ un predicado sobre un $ y:: \mathcal{T}$\\
\begin{flalign*}
& \forall cb\_params_1\ . \mathcal{P}(CB_1)\  \land \cdots \land\ \forall cb\_params_n\ . \ \mathcal{P}(CB_n)\\
& \forall cr\_params_1\ . \forall x ::\mathcal{T}\ . (\mathcal{P}(x) \implies \mathcal{P}(CR_1 ... x ....))\\
& \hspace{40pt}\vdots\hspace{95pt}\vdots \\
& \forall cr\_params_m\ . \forall x ::\mathcal{T}\ . (\mathcal{P}(x) \implies \mathcal{P}(CR_m ... x ....))
\end{flalign*}
\centering\rule{250pt}{0.4pt}
$$\vdash \forall x::\mathcal{T}\ .\ \mathcal{P}(x)$$
\end{samepage}

\section{Lemas de generación}
Todos los tipos inductivos tienen un lema de generación, viene de la noción de como se construyen los tipos, si tenemos alguna expresión
$$...E ...= ...$$
y sabemos que $E :: \mathcal{T}$ donde $\text{data }\mathcal{T}= A\ |\ B$ entonces sabemos que $E$ solo puede haber sido construído como $A$ o $B$.

En general tenemos que para algún tipo inductivo cualquiera
\begin{haskcode}
data Ttype a b = Nil | Const1 a | Const2 b | Const3 a b
\end{haskcode}
Tenemos que algúna expresión $E :: \text{Ttype}$, $E$ solo puede haber sido construido con uno de esos constructores, formalmente

si $e :: \text{Ttype}$ entonces

$\exists x :: a\ .\exists y :: b\ . (E = \text{Nil})\ \lor\ (E = \text{Const1 x}) \lor\ (E = \text{Const2 y}) \lor\ (E = \text{Const3 x y})$

\section{Principio de extensionalidad funcional}
Desde un punto de vista \textbf{intensional} dos valores son iguales si están construidos de la misma manera. Desde el punto de vista \textbf{extensional} dos valores son iguales so son indistinguibles al observador.

Normalmente pensamos que si $f = g$ entonces $\forall x :: a\ .f\ x = g\ x$ pensandolo de forma inversa hariamos el principio de extensionalidad funcional que es la siguiente implicación:

si $\forall x :: a\ .f\ x = g\ x$ entonces $f = g$

[TODO: falta isomorfismo de tipos]
[TODO: falta ejemplos de generalizacion de predicados de inductivos]
[TODO: Lemas auxiliares]

\section{Sistemas deductivos}

\end{document}
